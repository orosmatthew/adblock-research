\documentclass[sigsmall]{acmart}

\usepackage{graphicx}

\title{Summary of \emph{How Effective Are Commonly-Used Ad-Blockers?}}
\author{Matthew Oros, Sonny Smith, Michael Terekhov, Dennis Ulichney}

\setcopyright{none}
\settopmatter{printacmref=false}

\begin{document}
\maketitle


\section*{Abstract}
With the development of the internet web pages, online advertisements have also developed. Online advertisements can help find products that a user has been looking for, but they often ruin user experiences. Online advertisements might take up the whole space of a web page, and for mobile users, online advertisements might even make the usage of the page impossible. This research focuses on the software that removes such advertisements from web pages. The research paper includes an introduction, breakdown/related work, experiment design, results, and conclusion/future work.  \cite{}

\section*{Introduction}
The services offered online are mostly free and the most common way to monetize them is through online advertising. Online ads can be helpful, but most users find them to be one of the most annoying things on the internet. Most internet users want to minimize the number of ads they see and various ad-blocking software can be used to achieve this. The easiest and most popular way to block ads is to use browser extensions. An advanced ad-blocking method is to use specialized software that blocks incoming ad domains. Advanced software is considered more reliable, which means it blocks more ads. The purpose of this study is to find out how effective ad-blocking extensions are. This study focuses on the most popular browser ad blockers and their effectiveness compared to Pi-Hole software (the advanced software used for this study). Ad-blocking extensions that were utilized are AdBlock Plus, AdBlock, uBlock Origin, Privacy Badger, and Ghostery. Internet browsers that were utilized to test the software are Chrome, Microsoft Edge, and Firefox.


\section*{Background/Related Work}


\section*{Experiment Design}

\section*{Results}
The number of domains and ads blocked was first gathered from the blockers while in use and then gathered from proxy logs for comparison. The results of the first test found that uBlockOrigin detected and blocked the most domains on average followed by Ghostery and PrivacyBadger. uBlockOrigin blocked over 50\% more domains on average than the other blockers as shown in figure 1a. Although PrivacyBadger is not marketed as an adblocker, it blocked more domains on average than Adblock and Adblock Plus. 
AdBlock and AdBlock Plus blocked the lowest number of domains on average.   
The second test found that overall there were more domains and ads blocked by each blocker on average than their icon reports as shown in figure 1b. Similarly to the first test, uBlockOrigin was found to block the most domains due to the fewest number of domains and ads being detected. With this test there was an inverse relationship with the blocker's performance and the number of domains detected and blocked. The fewer the amount of total domains and ads detected is, the better the blocker. uBlockOrigin had less than 200 total domains to block while the other blockers had more than 400 total domains which indicates that uBlockOrigin also performed 50\% better than the other blockers in this test 
\begin{figure}[h]
  \includegraphics[scale = 0.75]{Edit3.png}
  \caption{The first graph(a) compares the total number of detected domains to the number ad-domains blocked. Smaller bars indicate more domains blocked.  The second graph(b) shows the average number of ad-domains reported by each adblocker.}
  \label{fig:graph1ab}
\end{figure}

\section*{Conclusion/Future Work}


\bibliographystyle{acm}
\bibliography{myBib}
\end{document}